\documentclass[12pt]{article}
\usepackage{natbib}
\usepackage{hyperref}
\usepackage{graphicx}
\usepackage{subcaption}
\usepackage{amssymb,amsmath,amsthm}
\usepackage{xcolor}
\usepackage{xspace}
\usepackage[nameinlink,capitalize]{cleveref}
\usepackage{cleveref}
\usepackage[margin=1in]{geometry}
\usepackage{lineno}\renewcommand\thelinenumber{\color{gray}\arabic{linenumber}}
\usepackage{pdflscape}
\usepackage{xspace}
\usepackage{array}

\newcolumntype{L}[1]{>{\raggedright\let\newline\\\arraybackslash\hspace{0pt}}m{#1}}
\newcolumntype{C}[1]{>{\centering\let\newline\\\arraybackslash\hspace{0pt}}m{#1}}
\newcolumntype{R}[1]{>{\raggedleft\let\newline\\\arraybackslash\hspace{0pt}}m{#1}}

\newcommand{\comment}{\showcomment}
\newcommand{\showcomment}[3]{\textcolor{#1}{\textbf{[#2: }\textsl{#3}\textbf{]}}}
\newcommand{\nocomment}[3]{}
\newcommand{\pkg}[1]{\textsf{#1}}  %{\texttt{#1}}

\newcommand{\ali}[1]{\comment{magenta}{Ali}{#1}}
\newcommand{\bmb}[1]{\comment{red}{BMB}{#1}}
\newcommand{\todo}[1]{\comment{red}{TODO}{#1}}

\theoremstyle{definition} % amsthm only
\newtheorem{proposition}{Proposition}
\newtheorem{theorem}{Theorem}
 
\bibliographystyle{apalike}

\title{Centrality: A Curve-Based Statistical Discription of an Ensemble}
\author{Ali Gharouni, Ben Bolker}
\begin{document}
\maketitle
\linenumbers

% %%%%%%%%%%%%%%%%%%%%%%%%%%%%%%%%%%%%%%%
\section{Abstract}
This is a comentary work motivated by \cite{juul2021fixed}'s work in which a few useful ideas of the concept of the central set out of an ensemble of epidemic curves were presented. In the present work we provide alternative curved-based statistics approaches to approximate the most central set which represents the central 50\% of the ensemble. In particular, we use three ranking methods of the curves; (i) the well-established, sampling-based, fast and robust functional boxplot, (ii) computing all pairwise distances (eg, $\ell_2$ norm) between the curves and ranking based on median-like distances, and (iii) a multivariate generalization of ranking the curves by using Mahalanobis distance among features of interest. We discuss and compare our results with the results presented by \cite{juul2021fixed}.      

\section{Introduction}

\cite{juul2021fixed} presented the following useful curve-based alternative methods to the fixed-time statistics of epidemic curve ensembles; (i) all-or-nothing ranking method (presented in their Fig.2b,c), (ii) weighted ranking (presented in their Fig.2d) and (iii) according to some feature of interest, e.g. the projected peak value (presented in their Fig.2e). While (i) and (ii) are sampling-based approach, (iii) is featur-based ranking approach. They cautioned that standard pointwise (fixed-time) averages may not be appropriate to summerize ensembles of epidemic curves. Particularly, they discussed that miscapturing key features of an epidemic such as the peak numbers of infections, the time of the peak, etc. may result in obscuring the forcast process. While Juul et al.'s implimented their methodology in sampling-based functional ranking and establishing the most 'central set' of an ensemble, there is a deep existing literature in functional depth-boxplot and centrality metrics for high dimension data \citep{fraiman2001trimmed, lopez2007depth, lopez2009concept, sun2011functional,sun2012exact}. It appears that \cite{juul2021fixed} were aware of the functional boxplot concept, since they cited the work by \cite{sun2011functional}. However, the reason/potential advantage of aparting from the standard functional boxplot framework remains unclear \ali{to be toned down!!}.  

The idea of the functional boxplot goes back farther to the notion of depth. It was first introduced for multivariate data in an attempt to generalize the ideas of order statistics, ranks and medians into higher dimensions (see for e.g., \cite{mahalanobis1936generalized,tukey1975mathematics}). The notion of depth was extended for functional data by \citep{lopez2009concept}. They introduced the concept of band depth (BD) for ranking a sample of functional data from the center outward. This ordeing enables us to define the functional quantiles, centrality or outlyingness of an observation. Further, the classical boxplot for univariate data was extended to the functional boxplots and adjusted functional boxplots as a visualization tool \citep{sun2011functional,sun2012adjusted}. 

The optimization of the computational cost of the functional band depth has been studied by \citep{sun2012exact}. The computational cost is determined by the sample size of the ensumble $n$ and the number of curves $j$ defining a band ($2\leq j \leq J$). \cite{sun2012exact} argued the robustness and sensiblity of  small value of $J$. In particular $J=2$ is preferable for large datasets. It is notable that in \cite{juul2021fixed}'s samling-based approach, i.e., methods (i) and (ii), $n=500$ and $J=50$ (in their paper: $N_{\rm{curves}}=50$). Random samples were uniformly drawn $N_{\rm{samples}}=100$ times, the centrality scores of all curves in the ensemble were updated based on the sample, and the most central curves are the ones with their scores above the 50\% percentile of scores. Here, we use the functional boxplot, specifically fda() in \pkg{roahd} package \citep{roahd} in R \citep{R}, with the choice of modified band depth (MBD) to break ties which is based on the fast algorithm proposed by \cite{sun2012exact}. Thus, by using \cite{juul2021fixed}'s dataset, ${500}\choose{2}$ samples determine the centrality scores in the fds().


\cite{juul2021fixed} also provided an example of functional ranking according to a some feature of interest, maximum values of newly hospitalized cases in a single day (Fig.~2e). The multivariate generalization of this approach can be implemented by using an appropriate metric among features. In particular, we considered the set of all vectors, in which a curve-spesific vector contains features of interest. We used the Mahalanobis distance \citep{mahalanobis1936generalized} to rank these feature vectors. The envelope of the most central curves includes the curves with the rank within the 50\% quantile. The Mahalanobis distance provides a measure of similarity between multivariate data and uses covariance information between features to weight the contributions to the distance. The Euclidean distance, on the other hand, in essence gives excess weight to variables (features) that are highly correlated and gives additional weight to variables that have similar information. The Mahalanobis distance gives less weight to those variables that have high variance and to those variables that have high correlation, so that other feature variables with lower correlations can contribute to the distance. The Mahalanobis distance has been used in pattern recognition for several decades,
including species identification in zoology \citep{robinson1975geographical}, diagnostic validity in neurology \citep{john1988neurometrics}. The epidemic features of ineterst that we considered here include the peak value of newly infections, time of the peak, final size of the epidemic, epidemic duration and the initial growth rate of the infection. Note that the choice of the epidemic features depends on the objectives of the study (see for e.g., \cite{probert2016decision}).

We also used Euclidean L2 norm, which gives the area between the two curves, on the set of 500 curves used by \cite{juul2021fixed}. The rank of each curve is then defined as the median distance of the curve with respect to the whole ensemble. The envelope of the most central curves includes the curves with the rank within the 50\% quantile.

\ali{working}
Here, the ``centroid'' of a set of functions (ensemble) represents the central 50\% of the ensemble.



Plot the envelope containing the most central curves, i.e. the pointwise min/max curves of a subset of curves with the lowest sum (or sum of squares) of distances to the other curves.

Note that the rankings for sum of distances to all other curves, mean-like, and sum of squared distances, median-like, are not quite identical. We'll use the 'median-like' criterion.


\section{Results}

\begin{figure}[h!]
\centering
\includegraphics[width=\linewidth]{scripts/cent_plot.pdf}
\caption{Comparison of alternative methods of functional boxplots.}\label{p.a}
\end{figure}


\bibliography{./AliMac}
\end{document}
