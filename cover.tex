\documentclass[12pt,letterpaper]{letter}

%% layout
\usepackage[margin=1.25in]{geometry}
\usepackage{multicol}

%% colours
\usepackage{xcolor}
\definecolor{linkcolor}{HTML}{0f87ff} % green
\definecolor{irena}{HTML}{c9657d} % blush
\definecolor{de}{HTML}{5681c4} % blue

\usepackage[colorlinks=true, allcolors=linkcolor]{hyperref}

%% date format
\usepackage{datetime}
\newdateformat{mydate}{\twodigit{\THEDAY}{ }\shortmonthname[\THEMONTH] \THEYEAR}

%% comment macros
\newcommand{\ali}[1]{\textcolor{irena}{$\langle$\textbf{AG:} #1$\rangle$}}

%% journal name macro
\newcommand{\journalname}{\emph{Nature Physics}}

\setlength{\parskip}{0em}

\begin{document}

\begin{multicols}{2}
\footnotesize
\begin{flushleft}

% Dr. Natalie Pafitis
% 
% Editor, \journalname{}

\vfill

{\normalsize \mydate
\today}
\end{flushleft}

\columnbreak

\begin{flushright}
Ali Gharouni

Postdoctoral Fellow 

Department of Mathematics and Statistics,

McMaster University,

Hamilton, ON, Canada, L8S 4K1

\href{mailto:aligharouni@gmail.com}{aligharouni@gmail.com}
\end{flushright}

\end{multicols}

\setlength{\parskip}{1em}
\thispagestyle{empty}

\vspace{-1em}

Dear Editor,

We are pleased to submit an original commentary article entitled
``Comment on Juul et al. (2021) (?)'' for your consideration for publication in \journalname{}. This is a minor commentary on the paper entitled: "Fixed-time descriptive statistics underestimate extremes of epidemic curve ensembles", Juul et al. 2021 (Nature Physics 17 (1): 5-8 doi: 10.1038/s41567-020-01121-y). It is notable we have contacted the authors and sent them a copy of our commentary. The authors have responded to us and they seem happy with our comments. 

In particular, our research article accomplishes the following.
we provide alternative, and more principled, curved-based statistics approaches to approximate the most central set which represents the central 50\% of the ensemble. In particular, we use two functional ranking methods; (1) the sampling-based, fast and robust functional boxplot, and (2) a multivariate generalization of ranking the curves by using Mahalanobis distance among features of interest. We apply our methods on the dataset of the original paper and compare our results with theirs.

All authors have approved the manuscript for submission and declare no competing interests. A preprint of this work has been posted to medRxiv (link \ali{do we need to do this?}). The manuscript has not been published or submitted for publication elsewhere. Also we have not had any prior discussions with a \journalname{} editor about the work described in the manuscript. 

Thank you very much for your consideration,

% \begin{multicols}{1}
\begin{flushleft}
\footnotesize

Ali Gharouni, PhD
\setlength{\parskip}{0em}

Postdoctoral Fellow, 

Department of Mathematics and Statistics, McMaster University

\vspace{1em}

% \vfill\null
% \columnbreak
Benjamin M.\ Bolker, PhD

Professor, Departments of Mathematics \& Statistics and Biology, 

McMaster University

Member, Michael G. DeGroote Institute for Infectious Disease Research, McMaster University
\end{flushleft}
% \end{multicols}

\thispagestyle{empty}

\end{document}
