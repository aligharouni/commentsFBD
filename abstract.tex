\documentclass[12pt]{article}

\usepackage{graphicx}
\usepackage{natbib}
\usepackage{hyperref}
\usepackage{grffile}
\usepackage{graphicx}
\usepackage{subcaption}
\usepackage{amssymb,amsmath}
\usepackage{xcolor}
\usepackage{xspace}
%

\begin{document}
% \section{Abstract}
This is a commentary work motivated by Juul et al. (2021)’s work in which a few useful ideas
of the concept of the central set out of an ensemble of epidemic curves were presented. In the present work we provide an alternative, and more principled, curved-based statistics approaches to approximate the most central set which represents the central 90\% of the ensemble. In particular, we use two functional ranking methods; (1) the sampling-based, fast and robust functional boxplot, and (2) a multivariate generalization of ranking the curves by using Mahalanobis distance among features of interest. We apply our methods on Juul et al. (2021)’s dataset and compare our results with theirs.

\end{document}
