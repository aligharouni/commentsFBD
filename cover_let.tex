\documentclass[11pt,a4paper,roman]{moderncv}        
\moderncvstyle{classic}                            
\moderncvcolor{green}
\usepackage[utf8]{inputenc}
\usepackage[scale=0.75]{geometry}
\usepackage[spanish,english,es-lcroman]{babel}
% personal data
\name{}{Ali Gharouni \& Benjamin M. Bolker}
\title{Cover Letter} 
\address{Department of Mathematics \& Statistics}{McMaster University, Hamilton, Canada}
\email{aligharouni@gmail.com}  


\begin{document}
\recipient{Dr. Jonas L. Juul}{
Center for Applied Mathematics, \\
Cornell University}
\date{\today}
\opening{Dear Dr. Juul,}

\closing{Please find enclosed our Matters Arising article. If you require any additional information, please do not hesitate to contact me. I look forward to hearing from you.
Thank you very much for your consideration.}

\makelettertitle

% \noindent 
I am writing to inform you that Dr. B. M. Bolker and I have been working on a Matters Arising article following your usefull article, Juul et al. 2021 (Nature Physics 17 (1): 5-8 doi: 10.1038/s41567-020-01121-y).
We would like to share our work with you and the authors of the original paper and request feedback before submitting it to Nature Physics.
This is not a criticism of the original paper, but makes a few points that were glossed over in the original. Spesifically, 
we provide alternative, and more principled, curved-based statistics approaches to approximate the most central set which represents the central 50\% of the ensemble. In particular, we use two functional ranking methods; (1) the sampling-based, fast and robust functional boxplot, and (2) a multivariate generalization of ranking the curves by using Mahalanobis distance among features of interest. We apply our methods on the dataset of the original paper and compare our results with theirs.

\vspace{0.5cm}
\makeletterclosing

\end{document}